% 导言区
\documentclass[12pt]{article} %全局字体大小(一般只有10、11、12pt)

\usepackage{ctex}

\newcommand{\myfont}{\textit{\textbf{\textsf{Fancy Text}}}}


% 正文区(文稿区)
\begin{document}
	% 字体族的设置(罗马字体、无衬线字体、打印机字体)
	\textrm{Roman Family} \textsf{Sans Serif Family} \texttt{Typewriter Family}
	
	{\rmfamily Roman Family} {\sffamily Sans Serif Family} {\ttfamily Typewriter Family}
	
	% 字体系列设置(粗细、宽度)
	\textmd{Medium Series} \textbf{Boldface Series}
	
	{\mdseries Medium Series} {\bfseries Boldface Series}
	
	
	% 字体形状设置(直立、斜体、伪斜体、小型大写)
	\textup{Upright Shape} \textit{Italic Shape}
	\textsl{Slanted Shape} \textsc{Small Caps Shape}
	
	{\upshape Upright Shape} {\itshape Italic Shape} {\slshape Slanted Shape} {\scshape Small Caps Shape}
	
	% 中文字体设置(要是用ctex宏包)
	{\songti 宋体} \quad {\heiti 黑体} \quad {\fangsong 仿宋} \quad{\kaishu 楷书}
	
	中文字体的\textbf{粗体}与\textit{斜体}
	
	% 字体大小设置
	{\tiny			Hello}\\
	{\scriptsize	Hello}\\
	{\footnotesize	Hello}\\
	{\small			Hello}\\
	{\normalsize	Hello}\\ %全局大小默认设置
	{\large			Hello}\\
	{\Large			Hello}\\
	{\LARGE			Hello}\\
	{\huge			Hello}\\
	{\Huge			Hello}\\
	
	
	% 中文字号设置命令
	\zihao{5} 你好!\\
	\zihao{-0} 你好!% 小初号
	
	% Latex的思想是内容与格式分离,建议新增newcommand。
	\myfont
	
	
	
\end{document}