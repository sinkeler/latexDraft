% 导言区
\documentclass{ctexart}

%\usepackage{ctex}
\usepackage{amsmath}
\newcommand{\adots}{\mathinner{\mkern2mu\raisebox{0.1em}{.}\mkern2mu\raisebox{0.4em}{.}\mkern2mu\raisebox{0.7em}{.}\mkern1mu}}


% 文稿区(正文区)
\begin{document}
	\[
	\begin{matrix} % 需要引入amsmath宏包
		0 & 1 \\
		1 & 0
	\end{matrix}
	\] % 注,\[和\]之间不能有空行
	
	% pmatrix环境
	\[
	\begin{pmatrix} % 用于在矩阵两端加小括号(parenthesis)
		0 & i \\
		i & 0
	\end{pmatrix} \qquad
	\]
	
	% bmatrix环境
	\[
	\begin{bmatrix} % 用于在矩阵两端加中括号(brackets)
		0 & i \\
		i & 0
	\end{bmatrix} \qquad
	\]

	% vmatrix环境
	\[
	\begin{vmatrix} % 用于在矩阵两端加单竖线(verticle lines)
		0 & i \\
		i & 0
	\end{vmatrix} \qquad
	\]
	
	% 可以使用上下标
	\[
	A = \begin{pmatrix} % 可以使用tab键,便于阅读源码
		a_{11}^2 	& a_{12}^2 	& a_{13}^2 \\
		0 			& a_{22}^2 	& a_{13}^2 \\
		0 			& 0 		& a_{33}
	\end{pmatrix}
	\]
	
	% 常用省略号:\dots(普通省略号)、\vdots(竖排省略号)、\ddots(对角线省略号)
	\[
	A = \begin{bmatrix} % 可以使用tab键,便于阅读源码
		a_{11} 		& \dots 		& a_{1n} \\
		 			& \ddots 		& \vdots \\
		0 			&  				& a_{nn}
	\end{bmatrix}_{n \times n}
	\]
	
	% 反对角线省略号\adots无法显示(averse),需要自定义newcommand
	\[
	A = \begin{bmatrix} % 可以使用tab键,便于阅读源码
		a_{11} 		& \dots 		& a_{1n} \\
		\adots		& \ddots 		& \vdots \\
		0 			&  				& a_{nn}
	\end{bmatrix}_{n \times n}
	\]
	
	% 分块矩阵,以下写法便于阅读
	\[
	\begin{pmatrix}
		\begin{matrix}
			1 & 0 \\
			0 & 1
		\end{matrix} 	& 	\text{\Large 0} \\
		\text{\Large 0} & 	\begin{matrix}
								1 & 0 \\
								0 & -1
						  	\end{matrix}
	\end{pmatrix}
	\]
	
	% 分块矩阵,也可以这么写
	\[
	\begin{pmatrix}
		\begin{matrix}	1 & 0 \\0 & 1	\end{matrix} 	
		& 	\text{\Large 0} \\
		\text{\Large 0} & 	\begin{matrix}
			1 & 0 \\0 & -1	\end{matrix}
	\end{pmatrix}
	\]
	
	% 分块矩阵,注意,如果不适用\text命令,则结果有差别
	\[
	\begin{pmatrix}
		\begin{matrix}
			1 & 0 \\
			0 & 1
		\end{matrix} 	& 	\Large 0 \\
		\text{\Large 0} & 	\begin{matrix}
			1 & 0 \\
			0 & -1
		\end{matrix}
	\end{pmatrix}
	\]
	
	% 三角矩阵
	\[
	\begin{pmatrix}
		a_{11}			& a_{12}							& \cdots	& a_{1n} \\
						& a_{22}							& \cdots	& a_{2n} \\
						&									& \ddots	& \vdots \\
		\multicolumn{2}{c}{\raisebox{1.3ex}[0pt]{\Huge 0}}	&			& a_{nn} % \multicolumn合并多列,\raisebox命令调整高度
	\end{pmatrix}
	\]
	
	% 跨列的省略号:\hdotsfor{列数}
	\[
	\begin{pmatrix}
		1	& \frac 12	& \dots	& \frac 1n 	\\
		\hdotsfor{4}							\\
		m	& \frac m2	& \dots	& \frac mn
	\end{pmatrix}
	\]
	
	% 行内小矩阵(smallmatrix)环境
	复数 $z=(x,y)$ 也可用矩阵
	\begin{math}
		\left(
			\begin{smallmatrix}
				x	& -y	\\
				y	& x
			\end{smallmatrix}
		\right)	
	\end{math}来表示
	
	% array环境(类似于表格环境tabular)
	\[
	\begin{array}{r|r} 					% 与tabular环境类似,r、l、c用来制定位置格式
		\frac12		& 0				\\	% 注意,\frac可以直接跟数字,但不能直接跟字母,这是一种简写
		\hline							% \hline产生横线
		0			& -\frac abc	\\
	\end{array}
	\]
	
	% 用array环境排版更为复杂的矩阵
	\[
	% @{<内容>}-添加任意内容,不占表项计数
	% 此处添加一个负值空白,表示向左移-5pt的距离,说的是p左边到大矩阵有括号的距离
	\begin{array}{c@{\hspace{-5pt}}l} % c表示第一列居中对齐,l表示最后一列居左对齐
		% 第1行,第1列
		\left(
			\begin{array}{ccc|ccc}
				a							& \cdots	& a			& b			& \cdots	& b			\\
											& \ddots	& \vdots	& \vdots	& \dots					\\
											&			& a			& b									\\
				\hline
											&			&			& c			& \cdots	& c			\\
											&			&			& \vdots	&			& \vdots	\\
				\multicolumn{3}{c|}{\raisebox{2ex}[0pt]{\Huge 0}}	& c			& \cdots	& c
			\end{array}
		\right)
		&
		% 第1行,第2列
		\begin{array}{l}
			% \left.仅表示与\right\}配对,什么都不输出
			\left.\rule{0mm}{7mm}\right\}p	\\
			\\
			\left.\rule{0mm}{7mm}\right\}q	\\
		\end{array}
		\\[-5pt]
		% 第2行,第1列
		\begin{array}{cc}
			\underbrace{\rule{17mm}{0mm}}_m	& \underbrace{\rule{17mm}{0mm}}_m
		\end{array}
		& % 第2行,第2列
	\end{array}
	\]
	

\end{document}