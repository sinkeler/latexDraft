% 导言区
\documentclass{ctexart}

\usepackage{amsmath}
\usepackage{amssymb}


% 文稿区(正文区)
\begin{document}
	\begin{gather} % gather环境实现多行环境排版(带编号)
		a + b = b + a \\ % 使用双反斜杠命令实现换行
		ab ba
	\end{gather}

	\begin{gather*}
		3 + 5 = 5 + 3 = 8 \\
		3 \times 5 = 5 \times 3
	\end{gather*}

	\begin{gather} % 可是在双反斜杠前使用 \notag 命令阻止编号
		3^2 + 4^2 = 5^2 \notag \\
		5^2 + 12^2 = 13^2 \notag \\
		a^2 + b^2 = c^2
	\end{gather}

	% align和align*环境(用&对齐)
	% 带编号
	\begin{align}
		x &= t + \cos t +_1 \\
		y &= 2\sin t
	\end{align}

	\begin{align*}
		x	&= t	& x	&= \cos t			& x	&= t		\\
		y	&= 2t	& y	&= \sin(t+1)		& y	&= \sin t
	\end{align*}

	% split环境,对齐采用align环境的方式,编号在两行中间,用&来连接两个公式
	\begin{equation} % 用equation排版,所以只有一个公式
		\begin{split}
			\cos 2x	& = \cos^2 x - \sin^2 x	\\
			& = 2\cos^2 x - 1
		\end{split}
	\end{equation}

	% cases 环境
	% 每行公式中使用 & 分隔为两部分
	% 通常表示值和后面的条件
	\begin{equation}
		D(x) = \begin{cases}
			1,	& \text{如果 } x \in \mathbb{Q};	\\ % 若使用中文排版数学公式则需使用\text,若不使用\text则无法显示中文字符
			0,	& \text{如果 } x \in \mathbb{R}\setminus\mathbb{Q}. % 需要amssymb宏包支持
		\end{cases}
	\end{equation}


\end{document}