% 导言区
\documentclass{article}

\usepackage{ctex} % 中文处理宏包
\usepackage{xltxtra} % 提供了针对XeTeX的改进并且加入了XeTeX的LOGO
\usepackage{texnames}
\usepackage{mflogo}

% 正文区(文稿区)
\begin{document}
	\section{空白符号}
	Are you wiser than      others? definitely no. in some ways, may it is true. What can you achieve? a luxurious house? a brilliant car? an admirable career? who knows?
	
	近年来,随着逆向工程和     三维重建技术的发展和应用,获取现实世界中物体的三维数据的方法越来越多的关注和研究,很多研究机构和商业公司都陆续推出了自己的三维重建系统。
	
	% 添加任意多的空格,也还是一个空格。
	% 空行分段,多个空行等同于1个
	% 自动缩进,绝对不能使用空格代替
	% 英文中多个空格处理为1个空格,中文中空格江北忽略
	% 汉子与其他字符的间距会自动由XeLaTeX处理
	% 禁止使用中文全角空格
	
	
	% 1em(当前字体中M的宽度)
	a\quad b
	
	% 2em
	a\qquad b
	
	% 约为1/6个em
	a\,b a\thinspace b
	
	% 0.5个em
	a\enspace b
	
	% 空格
	a\ b
	
	% 硬空格
	a~b
	
	% 1pc=12pt=4.218mm
	a\kern 1pc b
	
	a\kern -1em b % 负号刚好是反过来
	
	a\hskip 1em b
	
	a\hspace{35pt}b
	
	% 占位宽度
	a\hphantom{xyz}b
	
	% 弹性长度
	a\hfill b % 占满了整个空间
	
	
	\section{\LaTeX 控制符}
	
	
	\# \$ \% \{ \} \~{} \_{} \^{} \textbackslash
	\&
	% 注“\\”用于换行
	
	\section{排版符号}
	
	\S \P \dag \ddag \copyright \pounds
	
	
	\section{\TeX 标志符号}
	
	% 基本负号
	\TeX{} \LaTeX{} \LaTeXe{}
	\XeLaTeX{} % 注,需要引入xltxtra宏包
	
	
	% texnames宏包提供
	\AmSTeX{} \AmS-\LaTeX{} % 注意大小写
	\BibTeX{} \LuaTeX{}
	
	% mflogo宏包提供
	\METAFONT{} \MF{} \MP{}
	 
	\section{引号}
	
	` ' `` '' ``你好''
	
	\section{连字符}
	
	- -- ---
	
	\section{非英文字符}
	
	\oe \OE \ae \aa \AA \o \O \l \L \ss \SS !` ?`
	
	\section{重音符号(以o为例)}
	
	\`o \'o \^o \''o \~o \=o \.o \u{o} \v{o} \H{o} \r{o} \t{o} \b{o} \c{o} \d{o}	
\end{document}