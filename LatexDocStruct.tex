% 导言区
%\documentclass{article} %ctexbook, ctexrep

% \documentclass{ctexart}

\documentclass{ctexbook}




%\usepackage{ctex}

% ============= 设置标题的格式 ==========
% 注:latex代码不像编程语言,行与行之间不能有空行
\ctexset{
	section = {
		format+ = \zihao{-4} \heiti \raggedright,
		name = {,、},
		number = \chinese{section},
		beforeskip = 1.0ex plus 0.2ex minus .2ex,
		afterskip = 1.0ex plus 0.2ex minus .2ex,
		aftername = \hspace{0pt}
	},
	subsection = {
		format+ = \zihao{5} \heiti \raggedright,
		% name={\thesubsection、},
		name = {,、},
		number = \arabic{subsection},
		beforeskip = 1.0ex plus 0.2ex minus .2ex,
		afterskip = 1.0ex plus 0.2ex minus .2ex,
		aftername = \hspace{0pt}
	}
}

\begin{document}
%	\section{引言}
%	中国人口模式的转变发生于民国时期 关于民国的进步,我只讲两个过去人们比较忽略的问题。	一是人口模式。如前所述,传统时代人口的增减是王朝兴衰的显示器。
%	
%	中国人口模式的转变发生于民国时期 关于民国的进步,我只讲两个过去人们比较忽略的问题。	一是人口模式。如前所述,传统时代人口的增减是王朝兴衰的显示器。\par
%	中国人口模式的转变发生于民国时期 关于民国的进步,我只讲两个过去人们比较忽略的问题。	
%	\section{实验方法}
%	\section{实验结果}
%	\subsection{数据}
%	\subsection{图标}
%	\subsubsection{实验条件}
%	\subsubsection{实验过程}
%	\section{结论}
%	\section{致谢}
	
	
	
	
	%生成整个 文档的目录
	\tableofcontents
	
	\chapter{绪论} % 注意,需要讲documentclass更换为ctexbook
	\section{研究目的和意义}
	\section{国内外研究现状}
	\subsection{国外研究现状}
	\subsection{国内研究现状}
	\section{研究内容}
	\subsection{研究方法}
	\subsection{技术路线}
	
	\chapter{实验与结果分析}
	\section{引言}
	\section{实验方法}
	\section{实验结果}
	\subsection{数据}
	\subsection{图表}
	\subsubsection{实验条件}
	\subsubsection{实验过程}
	\subsection{结果分析}
	\section{结论}
	\section{致谢}

	
		
\end{document}

% 正文区(文稿区)