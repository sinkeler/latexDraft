% 导言区
\documentclass{ctexart}

% \usepackage{ctex}
% biblatex/biber
% 新的TEX参考文献排版引擎
% 样式文件(参考样式文件--bbx文件,引用样式文件--cbx文件)使用latex编写
% 支持根据本地化排版,如:
% 	biber -l zh__pinyin texfile,用于指定按拼音排序
%	biber -l zh_stroke texfile,用于按笔画排序
% 使用时,需要将[Build] → Default Bibilography Tool → Biber

%\usepackage[style=numeric,backend=biber]{biblatex}

%\usepackage[style=caspervector,backend=biber,utf8]{biblatex} % 使用比较优秀的caspervector样式,可从github上下载,默认是中英文混排
%\addbibresource{cnki2.bib} % 此处不能省略文件后缀,注,这个命令不能在{}中使用“,”来引用多个bib库文件

\usepackage[style=caspervector,backend=biber,utf8,sorting=cenyt]{biblatex} % c:中文、e:英文、n:作者姓名、t:文献标题、y:出版年份
%% 注意要排版两次产生正确的索引编号
\addbibresource{cnki2.bib} % 此处不能省略文件后缀,注,这个命令不能在{}中使用“,”来引用多个bib库文件

% 文稿区(正文区)
\begin{document}
	% 一次管理,多次引用
	无格式化引用\cite{__2020}
	
	带方括号的引用\parencite{_eomsoq4-n_2013}
	
	上标引用\supercite{_ti_3c_2_2020}
	
%	\printbibliography % 输出参考文件列表,默认title为英文Reference
	\nocite{*}
	\printbibliography[title={参考文献}]
	
	% 同样可以采用bat批处理文件实现编译
%	xelatex latexRefBibLaTeX
%	biber -l zh__pinyin latexRefBibLaTeX
%	xelatex latexRefBibLaTeX
%	xelatex latexRefBibLaTeX
%	del *.aux *.bbl *.bcf *.blg *.log *.xml
	
	
	
\end{document}