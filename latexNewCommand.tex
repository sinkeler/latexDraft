% 导言区
\documentclass{ctexart}

% \newcommand 定义新命令
% 命令只能由字母组成,不能以"\end"开头
% \newcommand<命令>[<参数个数>][<首参数默认值>]{<具体定义>}

% \newcommand 可以是简单字符串的替换,例如:
% 使用\PRC 相当于 People's Repulic of \emph{China} 这一串内容
\newcommand\PRC{People's Repulic of \emph{China}}

% \newcommand 也可以使用参数
% 参数个数可以是1 - 9,使用时用 #1、#2……表示
\newcommand\loves[2]{#1 喜欢 #2}
\newcommand\hateby[2]{#2 不受 #1 喜欢}

% \newcommand 的参数也可以使用默认值
% 指定参数个数的同时指定首个参数的默认值,那么这个命令的第一个参数就成为可选的参数
\newcommand\love[3][喜欢]{#2#1#3} % “喜欢”是默认参数

% \renewcommand - 重定义命令
% 与\newcommand 命令作用和用法相同,但只能用于已有命令的覆盖
% \renewcommand<命令>[<参数个数>][<首参数默认值>]{<具体定义>}
\renewcommand\abstractname{内容简介}

% 定义和重定义环境
% \newenvironment{<环境名称>}[<参数个数>][<首参数默认值>]{<环境前定义>}{<环境后定义>}
% \renewenvironment{<环境名称>}[<参数个数>][<首参数默认值>]{<环境前定义>}{<环境后定义>}

% 为book类中定义摘要(abstract)环境
\newenvironment{myabstract}[1][摘要]{\small
										\begin{center}
											\bfseries #1
										\end{center}
										\begin{quotation}} % 注意此处写法
									{\end{quotation}}

% 环境参数只有<环境前定义>中可以使用参数,
% <环境后定义>中不能再使用环境参数。
% 如果需要,可以把前面使用的参数保存到一个命令中,在后面使用。
\newenvironment{Quotation}[1]{\newcommand\quotesource{#1}
											\begin{quotation}}
												{\par\hfill---\textit{\quotesource}
											\end{quotation}}

% 文稿区(正文区)
\begin{document}
	\PRC
	
	\loves{猫儿}{鱼儿}
	\hateby{猫儿}{萝卜}
	
	\love{猫儿}{鱼儿} % 只提供两个参数,默认参数是“喜欢”,猫儿是#2,鱼儿是#3
	\love[最爱]{猫儿}{鱼儿} % “最爱”取代了默认参数“喜欢”
	
	\begin{abstract}
		这是一段摘要……
	\end{abstract}

	\begin{myabstract}
		这是一段自定义格式的摘要……
	\end{myabstract}

	% 或
	\begin{myabstract}[我的摘要]
		这是一段自定义格式的摘要……
	\end{myabstract}

	\begin{Quotation}{易$\cdot$乾}
		初九,潜龙勿用
	\end{Quotation}

	定义命令和环境是进行\LaTeX{}格式定制、达成内容与格式分离目标的利器。使用自定义的命令和环境把字体、自豪、缩进、对齐、间距等各种赶紧琐细的内容包装起来,赋以一个有意义的名字,可以使文档结构清晰、代码整洁、易于维护。在使用宏定义的功能时,要综合利用各种已有的命令、环境、变量等功能,事实上,前面所介绍的长度变量与盒子、字体字号等内容,大多并不直接出现在文档正文中,而主要都是用在实现各种结构化的宏定义里。
	
\end{document}
