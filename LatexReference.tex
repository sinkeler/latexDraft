% 导言区
\documentclass{ctexart}

% \bibliographystyle{plain} % plain, unsrt, alpha, abbrv
\bibliographystyle{unsrt} % plain, unsrt, alpha, abbrv

% \usepackage{natbib} % 通过natbib宏包可使用更多样式
% \bibliographystyle{plainnat} % plain, unsrt, alpha, abbrv


% 文稿区(正文区)
\begin{document}
	% 一次管理,一次使用
	% 参考文献格式:
	%\begin{thebibliography}{样本编号}
	%	\bibliography[记号]{引用标志}文献条目1
	%	\bibliography[记号]{引用标志}文献条目2
	%	……
	%\end{thebibliography}
	% 其中文献条目包括:作者,题目,出版社,年代,版本,页码等。
	% 引用时可以采用:\cite{引用标志1,引用标志2,……}
%	引用一篇文章\cite{article1}	引用一本书\cite{book1}等等
	
%	\begin{thebibliography}{99}
%		\bibitem{article1}陈立辉,苏伟,蔡川,陈晓云.\emph{基于LaTex的Web数学公式提取方法研究}[J]. 计算机科学. 2014(06)
%		\bibitem{book1}William H. Press, Saul A. Teukolsky, William T. Vetterling, Brain P. Flannery,\emph{Numerical Recipes 3rd Edition: The Art of Scientific Computing}Cambridge University Press, New York, 2007
%		\bibitem{latexGuide} Kopka Helmut, W. Daly Patrick, \emph{Guide to \LaTeX}, $4^{th}$ Edition. Available at \texttt{http://www.amazon.com}.
%		\bibitem{latexMath} Graetzer George, \emph{Math InTo \LaTeX}, BirkhAauser Boston; 3rd edition (June 22, 2000)
%	\end{thebibliography}

	% 以上过程较为繁琐,还未能实现“一次管理,多次使用”。
	% 可单独建立一个bib文件,用来制定文献关键字信息
%	@BOOK{mittelbach2004,
%		title = {The {{\LaTeX}} Companion},
%		publisher = {Addison-Wesley},
%		year = {2004},
%		author = {Frank Mittelbach and Michel Goossens},
%		series = {Tools and Techniques for Computer Typesetting},
%		address = {Boston},
%		edition = {Second}
%	}

	% 可以使用bib管理多个文献,利用【百度学术】、【Google学术】等,在相应的文献位置有【引用】功能,其中包含导出BibTex格式的功能,将其粘贴在bib文件中即可,如:
%	@BOOK{mittelbach2004,
%		title = {The {{\LaTeX}} Companion},
%		publisher = {Addison-Wesley},
%		year = {2004},
%		author = {Frank Mittelbach and Michel Goossens},
%		series = {Tools and Techniques for Computer Typesetting},
%		address = {Boston},
%		edition = {Second}
%	}
%	@article{刘伟2020疫情冲击下的,
%		title={疫情冲击下的2020年中国经济形势与政策选择},
%		author={刘伟 and 苏剑},
%		journal={社会科学研究},
%		volume={No.248},
%		number={03},
%		pages={28-35},
%		year={2020},
%	}


%	这是一个参考文献的引用:\cite{mittelbach2004},这是另一个引用:\cite{刘伟2020疫情冲击下的}
%	\bibliography{test} % 需要指定参考文献的排版样式\bibliographystyle{style}

	% 同样,手动维护bib数据库非常繁琐。
	
	% 可以搭配zotero工具获取cnki的引用,通过zotero导出bibtex条目 → cnki.bib
%	@article{_2020_2020,
%		title = {疫情冲击下的2020年中国经济形势与政策选择},
%		issn = {1000-4769},
%		url = {https://kns.cnki.net/kcms/detail/frame/list.aspx?dbcode=CJFD&filename=shyj202003003&dbname=CJFDLAST2020&RefType=1&vl=%mmd2B4Zm1wmuwRJRk0NahPfWv5UnFxmfrif3lj6RFFzFbMRsohMtzmY0k%mmd2BvR2a57hOlj},
%			abstract = {新冠肺炎疫情的出现是个"黑天鹅"事件。通过对疫情影响下的总供给、总需求及{CPI的测算},并综合自然走势和政策效果分析,预计2020年的中国经济增速为5.5\%-6.0\%,{CPI上涨率能够控制在}4.0\%以内。疫情对中国经济自然走势的负面影响很大,2020年增速目标的扩张性很强。为了实现目标,应采取扩张性需求管理为主、扩张性供给管理次之、扩张性市场环境管理为辅的政策组合。中国政府目前有足够的政策工具,如果坚持高目标,那么在相关的政策作用下,经济增速可能接近6.0\%;超过6.0\%的可能性不大,主要因为不需要那么高的增速;如果政府对其他政策目标考虑更多一些,则可能会容忍增速稍低,让政府有足够的余力应对其他...},
%			pages = {23--30},
%			number = {3},
%			journaltitle = {社会科学研究},
%			author = {刘, 伟 and 苏, 剑},
%			urldate = {2021-03-05},
%			date = {2020},
%			keywords = {2020年, 宏观调控, 经济增速, 扩张性, 疫情冲击, 政策结构, 政策组合, 自然走势, {CPI上涨率}},
%		}
%		
%		@article{__nodate,
%			title = {新冠肺炎疫情期间河北省国省道机动车交通量及大气污染物排放量变化分析},
%			issn = {0253-2468},
%			url = {https://kns.cnki.net/kcms/detail/detail.aspx?dbcode=CAPJ&dbname=CAPJDAY&filename=HJXX20210303002&v=jxwqIgFFRDsyQpCyKMeEwp1Vpoh5SwX4Qa9HigrTPO%25mmd2Brzp3PdtKIV5iVWPOKV038},
%				abstract = {本研究分别利用2019年、2020年第一季度河北省国省道交通调查站监测数据,计算了两年国省道机动车逐日交通量及大气污染物排放量,分析了新冠肺炎疫情期间国省道机动车大气污染物排放的变化情况.与2019年相比,2020年第一季度河北省国省道交通量同比下降38.1\%,单位公里{CO}、{VOCs}、{NO}\_x、{PM}\_(2.5)、{PM}\_(10)排放强度分别同比下降31.3\%、32.7\%、19.1\%、20.2\%、20.0\%.从不同公路类型来看,2020年第一季度,普通公路交通量持续下降,国家高速和省级高速交通量在3月出现回升,分别同比增长5.6\%、37.2\%,且货车增速高于客车.2020年春运期间,客车、货车总交...},
%				pages = {1--9},
%				journaltitle = {环境科学学报},
%				author = {孟, 琛琛 and 倪, 爽英 and 陆, 雅静 and 杜, 杰 and 安, 学文 and 宿, 文康 and 李, 浩坤},
%				urldate = {2021-03-05},
%				keywords = {大气污染物, 国省道, 河北省, 机动车, 新冠肺炎疫情, air pollutants, {COVID}-19 epidemic, Hebei Province, motor vehicles, national and provincial roads},
%			}
%			
%			@article{__2021,
%				title = {动物新型冠状病毒流行现状},
%				issn = {1005-944X},
%				url = {https://kns.cnki.net/kcms/detail/detail.aspx?dbcode=CAPJ&dbname=CAPJDAY&filename=ZGDW20210304009&v=2vpJqQNi66FRIXATD4%25mmd2B8hA7nKwDG%25mmd2FdO7s37iVmx6fTKaNV1aZFWB7UpklYAUgfWN},
%					abstract = {新型冠状病毒({SARS}-{CoV}-2)感染导致的新冠肺炎({COVID}-19)疫情自2019年12月以来,已在全球221个国家和地区传播,导致1亿多人感染,230多万人死亡。{SARS}-{CoV}-2除了对公共卫生产生极大影响之外,全球已有24个国家和地区报告470余起动物感染疫情,涉及犬、猫、水貂、狮子、老虎等多种动物。因此,在同一健康框架下,应高度关注动物感染{SARS}-{CoV}-2的早期预警。本文描述了动物感染{SARS}-{CoV}-2的国际流行特点,对其流行现状和感染动物种类进行了综合分析,评估了当前国内防控动物{SARS}-{CoV}-2感染面临的形势,并提出了针对性的防控建议,为国内{COVID}-19联防联控和...},
%					pages = {63--67},
%					number = {3},
%					journaltitle = {中国动物检疫},
%					author = {刘, 华雷},
%					urldate = {2021-03-05},
%					date = {2021},
%					keywords = {动物, 流行现状, 新型冠状病毒, animal, prevalence status, {SARS}-{CoV}-2},
%					file = {Full Text PDF:C\:\\Users\\木村\\Zotero\\storage\\3EV75ZZR\\刘 - 2021 - 动物新型冠状病毒流行现状.pdf:application/pdf},
%				}
%				
%				@article{_bgg-dsge_2021,
%					title = {{BGG}-{DSGE模型下罕见灾难风险宏观经济效应研究}——兼论新冠肺炎疫情的宏观经济影响},
%					volume = {61},
%					issn = {0257-2834},
%					url = {https://kns.cnki.net/kcms/detail/detail.aspx?dbcode=CJFD&dbname=CJFDAUTODAY&filename=JLDB202102013&v=Vb4DmadfUsRT%25mmd2BgTNcArN4L44uEj7a9YbtWt8iYjWR3MLGKGBdWaXRYxobZeCJ2mo},
%						abstract = {将罕见灾难风险引入{BGG}-{DSGE模型},风险类型扩展为全要素生产率灾难风险、资本灾难风险和消费灾难风险,分析罕见灾难风险的传导机制与宏观经济影响。研究发现,与真实经济周期框架下的分析结论不同,全要素生产率灾难风险使得我国的就业与企业外部融资状况有所改善,但会抑制产出与消费,产生通货膨胀压力;资本灾难风险短期会刺激产出和消费,但是存在通货紧缩倾向,且企业外部融资成本上升;消费灾难风险能够缓解企业融资压力,刺激投资,但产出与消费受损,就业压力增加。新冠肺炎疫情对消费、投资需求的初始冲击通过居民收入与消费需求间、劳动需求与投资水平间、资本品需求与企业财务状况间的循环作用机制,对消费、储蓄投资、产出、...},
%						pages = {116--127+237},
%						number = {2},
%						journaltitle = {吉林大学社会科学学报},
%						author = {史, 本叶 and 杨, 善然},
%						urldate = {2021-03-05},
%						date = {2021},
%						keywords = {贝叶斯估计, 罕见灾难风险, 宏观经济效应, 新冠肺炎疫情, Bayesian estimation, {BGG}-{DSGE}, {COVID}-19, macroeconomic effect, rare disaster risk},
%					}
%					
%					@online{noauthor__nodate,
%						title = {灾难性事件后目的地旅游业复苏路径分析——新冠肺炎疫情后武汉旅游发展 - 中国知网},
%						url = {https://kns.cnki.net/kcms/detail/detail.aspx?dbcode=CJFD&dbname=CJFDAUTODAY&filename=HBRW202103012&v=y8xyhFNXS%25mmd2BfJBj8lGXt3KjZf%25mmd2Fh8I0MVoji9ptPKwW4Pai2av9D%25mmd2Bi0kdKfZy%25mmd2FnxK%25mmd2F},
%							urldate = {2021-03-05},
%						}
	
	这是一个参考文献的引用:\cite{mittelbach2004},这是另一个引用:\cite{刘伟2020疫情冲击下的}
	
	这是一个来自知网的文献:\cite{_2020_2020}
	
	\nocite{*} % 输出所有未引用的参考文献,注意要经常清理过程文件,Tools → Clean Auxiliary Files
	\bibliography{test,cnki}
\end{document}